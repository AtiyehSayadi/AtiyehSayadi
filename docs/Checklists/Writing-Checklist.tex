\documentclass[12pt]{article}

\usepackage{enumitem}

\usepackage{amssymb}
\usepackage{amsfonts}
\usepackage{amsmath}

\usepackage{hyperref}
\hypersetup{colorlinks=true,
    linkcolor=blue,
    citecolor=blue,
    filecolor=blue,
    urlcolor=blue,
    unicode=false}
\urlstyle{same}

\newlist{todolist}{itemize}{2}
\setlist[todolist]{label=$\square$}
\usepackage{pifont}
\newcommand{\cmark}{\ding{51}}%
\newcommand{\xmark}{\ding{55}}%
\newcommand{\done}{\rlap{$\square$}{\raisebox{2pt}{\large\hspace{1pt}\cmark}}%
\hspace{-2.5pt}}
\newcommand{\wontfix}{\rlap{$\square$}{\large\hspace{1pt}\xmark}}

\begin{document}

\title{Writing Checklist}
\author{Spencer Smith}
\date{\today}

\maketitle

% Show an item is done by   \item[\done] Frame the problem
% Show an item will not be fixed by   \item[\wontfix] profit

The following checklist should be applied to the documents that you write.

\begin{itemize}
  
\item \LaTeX{} points
  \begin{todolist}
  \item Only tex file (and possibly pdf files, or image files) are under version
    control (\texttt{aux} files etc. are not under version control) (\texttt{.gitignore} can help with this)
    \item Opening and closing ``quotes'' are used (\verb|``quotes''|)
  \item Periods that do not end sentences are followed by only one space
    (\LaTeX{} inserts two by default): \verb|``I like Dr.\ Smith.''|, or for no
    linebreaks: \verb|``I like Dr.~Smith.''|
  \item Long names in math mode use either mathit or text, or equivalent:
    $coeff$ (\verb|$coeff$|) versus $\mathit{coeff}$ (\verb|$\mathit{coeff}$|)
    versus $\text{coeff}$ (\verb|$\text{coeff}$|).  (By default \LaTeX{}
    incorrectly thinks each letter is a separate variable and spaces
    accordingly.)
  \item The \texttt{Comments.tex} file is provided for ease of adding comments
  to a LaTeX file.
  \item Text lines should be 80 characters wide.  That is, the text has a hard-wrap at
    80 characters.  This is done to facilitate meaingful diffs between different
    commits.  (Some ideas on how to do this are given below.)
  \item Template comments (plt) should not be shown in the pdf version, either
    by removing them, or by turning them off.
  \item References and labels are used so that maintenance is feasible
  \item Cross-references between documents are used as appropriate
  \item BibTeX is used for generating bibliographic references
  \item \href{https://nhigham.com/2019/11/19/better-latex-tables-with-booktabs/}
  {booktabs} is used to generate tables, vertical and horizontal lines are
  minimized
  \end{todolist}

\item Structure
  \begin{todolist}
  \item There is always some text between section headings
  \item There aren't instances of only one subsection within a section
  \end{todolist}

\item Spelling, Grammar and attention to detail
  \begin{todolist}
  \item Each punctuation symbol (period, comma, semicolon, question mark,
    exclamation point) has no space before it.
  \item Opening parentheses (brackets) have a space before, closing parentheses
    have a space after the symbol.
  \item Parentheses (brackets) occur in pairs, one opening and one closing
  \item All sentences begin with a capital letter.
  \item Document is spell checked!
  \item Grammar has been checked (review, assign a team member an issue to review the grammar for each section).
  \item That and which are used correctly
    (\url{http://www.kentlaw.edu/academics/lrw/grinker/LwtaThat_Versus_Which.htm})
  \item Symbols used outside of a formula should be formatted the same way as
    they are in the equation.  For instance, when listing the variables in an
    equation, you should still use math mode for the symbols.
  \item Include a \texttt{.gitignore} file in your repo so that generated files
    are ignored by git.  More information is available on-line on
    \href{https://en.wikipedia.org/wiki/Hidden_file_and_hidden_directory}
    {Hidden files and hidden directories}.
  \item All hyperlinks work
  \item Every figure has a caption
  \item Every table has a heading
  \item Every figure is referred to by the text at some point
  \item Every table is referred to be the text at some point
  \item All acronyms are expanded on their first usage, using capitals to
    show the source of each letter in the acronym.  Defining the acronym only in
    a table at the beginning of the document is not enough.
  \item \href{https://www.scribendi.com/academy/articles/when_to_spell_out_numbers_in_writing.en.html?session_token=eyJ0aW1lIjoxNjY2MDY0ODcyOTQ4LCJob3N0Ijoid3d3LnNjcmliZW5kaS5jb20iLCJyZWZlcmVyIjoiaHR0cHM6Ly93d3cuc2NyaWJlbmRpLmNvbS9hY2FkZW15L2FydGljbGVzL3doZW5fdG9fc3BlbGxfb3V0X251bWJlcnNfaW5fd3JpdGluZy5lbi5odG1sIn0%3D} {All numbers from zero to ten should be written out as words.  Larger numbers should be written as numerals.}
  \end{todolist}

\item Avoid Low Information Content phrases 
    (\href{https://www.webpages.uidaho.edu/range357/extra-refs/empty-words.htm}{List
      of LIC phrases})

  \begin{todolist}
  \item ``in order to'' simplified to ``to''
  \item ...
  \end{todolist}

\item Writing style
  \begin{todolist}
  \item Avoid sentences that start with ``It.''
  \item Paragraphs are structured well (clear topic sentence, cohesive)
  \item Paragraphs are concise (not wordy)
  \end{todolist}
  
\item GitHub
  \begin{todolist}
  \item Proper GitHub conventions are followed (see below)
  \end{todolist}

\end{itemize}

\subsubsection*{Fixed Width \LaTeX{} Text}

Having the \LaTeX{} text at a fixed width (hard-wrap) is useful when the source
is under version control.  The fixed line lengths help with isolating the
changes between diffs.

Although the checklist mentions an 80 column width, any reasonable fixed width
is fine.

The hard-wrap shouldn't be done manually.  Most editors will have some facility
for fixed width.  In emacs it is called auto-fill.  Some advice from previous
and current students:

\begin{itemize}
\item In TEXMaker, you can do: User $>$ Run script $>$ hardwordwrap
\item Wrapping is easy in VSCode, Emacs, and Vim
\end{itemize}

\subsubsection*{Using GitHub Checklist}

\begin{itemize}
\item When closing an issue, include (where appropriate) the commit hash of the
  commit that addresses the issue
\item Make small commits (sometimes a commit will be changing only one line, or
  even just one word)
\item Make sure that all of the changes in a commit are related (you can change
  more than one file, but the changes should all be related)
\item You can easily link to other issues in your issue description or
  discussion comments by using the hash symbol followed by the number
  of that other issue
\item If your issue references a document or file in your repo (or elsewhere)
include a link to that document to make it easier for others to navigate your
issue
\item Make your issue as self-contained as possible.  Sometimes this will
involve including a screenshot or other graphic.
\item You can include a smiley face :smile:, if you want to ensure that your
  comments do not come across as more harsh than you intend
\item What needs to happen to close an issue should be clear
\item Close issues when they are done (the person assigned the issue is
  generally the person that closes the issue)
\item Most issues should be explicitly assigned to someone
\item 
  \href{https://gitlab.cas.mcmaster.ca/smiths/se2aa4_cs2me3/-/blob/master/FAQ/GitAdvice.txt}
  {Advice from Emily Horsman} on git commits
\end{itemize}

There are many other
\href{https://gitlab.cas.mcmaster.ca/SEforSC/se4sc/-/wikis/Advice-and-Checklists-for-Repos-(including-a-list-of-recommended-artifacts)}{checklists}
available for scientific computing (research) software.  Google can help find
even more checklists.

\end{document}
