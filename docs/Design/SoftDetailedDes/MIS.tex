\documentclass[12pt, titlepage]{article}

\usepackage{amsmath, mathtools}

\usepackage[round]{natbib}
\usepackage{amsfonts}
\usepackage{amssymb}
\usepackage{graphicx}
\usepackage{colortbl}
\usepackage{xr}
\usepackage{hyperref}
\usepackage{longtable}
\usepackage{xfrac}
\usepackage{tabularx}
\usepackage{float}
\usepackage{siunitx}
\usepackage{booktabs}
\usepackage{multirow}
\usepackage[section]{placeins}
\usepackage{caption}
\usepackage{fullpage}

\hypersetup{
bookmarks=true,     % show bookmarks bar?
colorlinks=true,       % false: boxed links; true: colored links
linkcolor=red,          % color of internal links (change box color with linkbordercolor)
citecolor=blue,      % color of links to bibliography
filecolor=magenta,  % color of file links
urlcolor=cyan          % color of external links
}

\usepackage{array}

\externaldocument{../../SRS/SRS}

\input{../../Comments}
\input{../../Common}

\begin{document}

\title{Module Interface Specification for Centrality in Graphs}

\author{Atiyeh Sayadi}

\date{\today}

\maketitle

\pagenumbering{roman}

\section{Revision History}

\begin{tabularx}{\textwidth}{p{3cm}p{2cm}X}
\toprule {\bf Date} & {\bf Version} & {\bf Notes}\\
\midrule
Date 1 & 1.0 & Notes\\
Date 2 & 1.1 & Notes\\
\bottomrule
\end{tabularx}

~\newpage

\section{Symbols, Abbreviations and Acronyms}

\renewcommand{\arraystretch}{1.2}
\begin{tabular}{l l} 
  \toprule		
  \textbf{symbol} & \textbf{description}\\
  \midrule 
  AC & Anticipated Change\\
CC & Closeness Centrality\\
  CIG& Centrality in Graphs\\
  DAG & Directed Acyclic Graph \\
DC & Degree Centrality\\
  M & Module \\
 n & Total number of nodes \\
  R & Requirement\\
  SRS & Software Requirements Specification\\
  UC & Unlikely Change \\
  \bottomrule
\end{tabular}\\

\newpage

\tableofcontents

\newpage

\pagenumbering{arabic}

\section{Introduction}

The following document details the Module Interface Specifications for the centrality in graphs project. As stated earlier in various documents such as the SRS, this project aims to measure the centrality of each node in an undirected graph using two methods: degree centrality and closeness centrality.

Complementary documents include the System Requirement Specifications
and Module Guide.  The full documentation and implementation can be
found at \url{https://github.com/AtiyehSayadi/Centrality-In-Graphs/blob/main/docs/ProblemStatementAndGoals/ProblemStatement.pdf}. 
\section{Notation}



The structure of the MIS for modules comes from \citet{HoffmanAndStrooper1995},
with the addition that template modules have been adapted from
\cite{GhezziEtAl2003}.  The mathematical notation comes from Chapter 3 of
\citet{HoffmanAndStrooper1995}.  For instance, the symbol := is used for a
multiple assignment statement and conditional rules follow the form $(c_1
\Rightarrow r_1 | c_2 \Rightarrow r_2 | ... | c_n \Rightarrow r_n )$.

The following table summarizes the primitive data types used by CIG. 

\begin{center}
\renewcommand{\arraystretch}{1.2}
\noindent 
\begin{tabular}{l l p{7.5cm}} 
\toprule 
\textbf{Data Type} & \textbf{Notation} & \textbf{Description}\\ 
\midrule
character & char & a single symbol or digit\\
integer & $\mathbb{Z}$ & a number without a fractional component in (-$\infty$, $\infty$) \\
natural number & $\mathbb{N}$ & a number without a fractional component in [1, $\infty$) \\
real & $\mathbb{R}$ & any number in (-$\infty$, $\infty$)\\
matrix&$m^{i \times j}$ & matrix with i rows and j columns\\
NodeT& ${\{char, \mathbb{N}\}}^+$& char, integer or both\\
EdgeT& tuple of (NodeT, NodeT)& -\\
Graph& set of EdgeT& -\\

\bottomrule
\end{tabular} 
\end{center}

\noindent
The specification of CIG uses some derived data types: matrix. A matrix is a collection of elements arranged in rows and columns within a rectangular array. Each element in the matrix can be any scalar value, such as real numbers, complex numbers, or variables. Also, graph is a collection of nodes and edges, where each node represents elements within the network, and the edges represent the connections between them. In addition, CIG uses functions, which
are defined by the data types of their inputs and outputs. Local functions are
described by giving their type signature followed by their specification.

\section{Module Decomposition}

The following table is taken directly from the Module Guide document for this project.

\begin{table}[h!]
\centering
\begin{tabular}{p{0.3\textwidth} p{0.6\textwidth}}
\toprule
\textbf{Level 1} & \textbf{Level 2}\\
\midrule

{Hardware-Hiding Module} & - \\
\midrule

{Behaviour-Hiding Module} & GUI \\ & File\\ &Graph\\
\midrule

\multirow{3}{0.3\textwidth}{Software Decision Module} & Matrix\\ & Output\\
\bottomrule

\end{tabular}
\caption{Module Hierarchy}
\label{TblMH}
\end{table}

\newpage

\section{MIS of File } \label{Module} 


\subsection{Module}
File , Matrix, Hardware-Hiding

\subsection{Uses}
N/A
\subsection{Syntax}

\subsubsection{Exported Types}
NodeT= ${\{char, \mathbb{N}\}}^+$
\subsubsection{Exported Access Programs}

\begin{center}
\begin{tabular}{p{2cm} p{2cm} p{2cm} p{4cm}}
\hline
\textbf{Name} & \textbf{In} & \textbf{Out} & \textbf{Exceptions} \\
\hline
\texttt{read\_file} & String & $NodeT^{n \times 2}$ & - \\
\hline
\end{tabular}
\end{center}

\subsection{Semantics}

\subsubsection{State Variables}

N/A

\subsubsection{Environment Variables}

file: A sequence of memory bits

\subsubsection{Assumptions}

N/A

\subsubsection{Access Routine Semantics}

\noindent \texttt{read\_file(s)}
\begin{itemize}
\item transition: - 
\item output:   $out:=\{\forall i:\mathbb{N}|i \in \{1..the\ number\ of\ distinguished\ elements\ in\ s\}\wedge \left(s1,s2\right)\in s: matrix\left [i,0\right]=s1\wedge matrix\left[i,1\right]=s2 \}$
\item exception: -  
\end{itemize}

%\wss{A module without environment variables or state variables is unlikely to
  %have a state transition.  In this case a state transition can only occur if
  %the module is changing the state of another module.}

%\wss{Modules rarely have both a transition and an output.  In most cases you
  %will have one or the other.}

\subsubsection{Local Functions}

N/A

\newpage

\section{MIS of Graph} \label{Module} 


\subsection{Module}
Graph

\subsection{Uses}
File, Matrix
\subsection{Syntax}

\subsubsection{Exported Types}
EdgeT= tuple of (start: NodeT, end: NodeT)\\
GraphT= set of EdgeT
\subsubsection{Exported Access Programs}
\begin{center}
\begin{tabular}{p{6cm} p{4cm} p{4cm} p{2cm}}
\hline
\textbf{Name} & \textbf{In} & \textbf{Out} & \textbf{Exceptions} \\
\hline
\texttt{build\_graph} & $NodeT^{n \times 2}$  & - & - \\
\texttt{exist\_edge} & EdgeT & $\mathbb{B}$  & - \\
\texttt{get\_degree} & NodeT & $\mathbb{N}$ & - \\
\texttt{get\_shortest\_path} & NodeT, NodeT & $\mathbb{N}$ & - \\
\texttt{degree\_centrality} & NodeT & $\mathbb{R}$ & - \\
\texttt{closeness\_centrality} & NodeT & $\mathbb{R}$ & - \\

\hline
\end{tabular}
\end{center}

\subsection{Semantics}

\subsubsection{State Variables}

nodes: set of NodeT\\
edges: GraphT


\subsubsection{Environment Variables}

N/A

\subsubsection{Assumptions}


{build\_graph} is called before any other access programs.



\subsubsection{Access Routine Semantics}

\noindent \texttt{build\_graph(g)}
\begin{itemize}
\item transition: $edges:= \{\left(s,e\right):  EdgeT| s,e \in g:\left(s,e\right)\}$, $nodes:= \{s: T| s \in g: s\}$
\item output: -
\item exception: -
\end{itemize}

\noindent \texttt{exist\_edge(s, e)}
\begin{itemize}
\item transition: -
\item output: $out:= \exists \{\left(s,e\right)\: EdgeT| \left(s,e\right) \in edges: s,e\}$
\item exception:-  
\end{itemize}

\noindent \texttt{get\_degree(n)}
\begin{itemize}
\item transition: -
\item output: $out:= Sum(\{i,n: NodeT|  \left(n,i\right) \in edges: 1\})$
\item exception: -
\end{itemize}

\noindent \texttt{get\_shortest\_path(s,n)}
\begin{itemize}
\item transition: -
\item output: $out:= \{path: sequence\ of\ NodeT|\ path=\{s..n\}: |path|\ is\ minimum\} $
\item exception: -   
\end{itemize}

\noindent \texttt{degree\_centrality(n)}
\begin{itemize}
\item transition: -
\item output: $out:=\{n: NodeT|\ n \in nodes : \frac{get\_degree(n)}{|nodes| - 1} \} $
\item exception: - 
\end{itemize}

\noindent \texttt{closeness\_centrality}
\begin{itemize}
\item transition: -
\item output: $out:=\{n: NodeT|\ n \in nodes : \frac{|nodes |- 1}{\sum_{for\ each\ j\ \in nodes}{get\_shortest\_path(n,j)}} \} $
\item exception: - 
\end{itemize}

\subsubsection{Local Functions}

N/A

\newpage

\section{MIS GUI} \label{Module} 


\subsection{Module}
GUI

\subsection{Uses}
Graph, Output, Hardware-Hiding
\subsection{Syntax}

\subsubsection{Exported Constants}
N/A
\subsubsection{Exported Access Programs}
\begin{center}
\begin{tabular}{|p{6cm}|p{2.5cm}|p{2.5cm}|p{4cm}|}
\hline
\textbf{Name} & \textbf{In} & \textbf{Out} & \textbf{Exceptions} \\
\hline
\texttt{degree\_centrality} & GraphT& - & - \\
\texttt{closeness\_centrality} & GraphT & - & - \\
\hline
\end{tabular}
\end{center}

\subsection{Semantics}

\subsubsection{State Variables}

N/A

\subsubsection{Environment Variables}

win: two-dimensional sequence of colored pixels

\subsubsection{Assumptions}

N/A

\subsubsection{Access Routine Semantics}

\noindent \texttt{degree\_centrality(g)}
\begin{itemize}
\item transition: Adjusting the pixels of the win for displaying the graph g
\item output: -
\item exception: N/A  
\end{itemize}

\noindent \texttt{closeness\_centrality(g)}
\begin{itemize}
\item transition: Adjusting the pixels of the win for displaying the graph g 
\item output: -
\item exception: N/A  
\end{itemize}

\subsubsection{Local Functions}

N/A

\newpage

\section{MIS of Output} \label{Module} 


\subsection{Module}
Output 

\subsection{Uses}
Hardware-Hiding
\subsection{Syntax}

\subsubsection{Exported Constants}
N/A
\subsubsection{Exported Access Programs}
\begin{center}
\begin{tabular}{p{6cm} p{4cm} p{4cm} p{2cm}}
\hline
\textbf{Name} & \textbf{In} & \textbf{Out} & \textbf{Exceptions} \\
\hline
\texttt{output} & string, $\mathbb{N}$,.. & GUI & - \\

\hline
\end{tabular}
\end{center}


\subsection{Semantics}

\subsubsection{State Variables}

N/A

\subsubsection{Environment Variables}

N/A

\subsubsection{Assumptions}

N/A
\newpage

\section{MIS of Matrix} \label{Module} 


\subsection{Module}
Matrix

\subsection{Uses}
N/A
\subsection{Syntax}

\subsubsection{Exported Constants}
N/A
\subsubsection{Exported Access Programs}
\begin{center}
\begin{tabular}{p{6cm} p{4cm} p{4cm} p{2cm}}
\hline
\textbf{Name} & \textbf{In} & \textbf{Out} & \textbf{Exceptions} \\
\hline
\texttt{Matrix} & string, $\mathbb{N}^{n \times n}$, list, ... & $\mathbb{N}^{n \times n}$, $\mathbb{N},..$  & - \\

\hline
\end{tabular}
\end{center}


\subsection{Semantics}

\subsubsection{State Variables}

N/A

\subsubsection{Environment Variables}

N/A

\subsubsection{Assumptions}

N/A




\bibliographystyle {plainnat}
\bibliography {../../../refs/References}


\section{Reflection}

The information in this section will be used to evaluate the team members on the
graduate attribute of Problem Analysis and Design.  Please answer the following questions:

\begin{enumerate}
  \item What are the limitations of your solution?  Put another way, given
  unlimited resources, what could you do to make the project better? (LO\_ProbSolutions)
  \item Give a brief overview of other design solutions you considered.  What
  are the benefits and tradeoffs of those other designs compared with the chosen
  design?  From all the potential options, why did you select the documented design?
  (LO\_Explores)
\end{enumerate}


\end{document}