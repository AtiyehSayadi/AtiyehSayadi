\documentclass[12pt, titlepage]{article}

\usepackage{amsmath, mathtools}

\usepackage[round]{natbib}
\usepackage{amsfonts}
\usepackage{amssymb}
\usepackage{graphicx}
\usepackage{colortbl}
\usepackage{xr}
\usepackage{hyperref}
\usepackage{longtable}
\usepackage{xfrac}
\usepackage{tabularx}
\usepackage{float}
\usepackage{siunitx}
\usepackage{booktabs}
\usepackage{multirow}
\usepackage[section]{placeins}
\usepackage{caption}
\usepackage{fullpage}

\hypersetup{
bookmarks=true,     % show bookmarks bar?
colorlinks=true,       % false: boxed links; true: colored links
linkcolor=red,          % color of internal links (change box color with linkbordercolor)
citecolor=blue,      % color of links to bibliography
filecolor=magenta,  % color of file links
urlcolor=cyan          % color of external links
}

\usepackage{array}

\externaldocument{../../SRS/SRS}

%% Comments

\usepackage{color}

\newif\ifcomments\commentstrue %displays comments
%\newif\ifcomments\commentsfalse %so that comments do not display

\ifcomments
\newcommand{\authornote}[3]{\textcolor{#1}{[#3 ---#2]}}
\newcommand{\todo}[1]{\textcolor{red}{[TODO: #1]}}
\else
\newcommand{\authornote}[3]{}
\newcommand{\todo}[1]{}
\fi

\newcommand{\wss}[1]{\authornote{blue}{SS}{#1}} 
\newcommand{\plt}[1]{\authornote{magenta}{TPLT}{#1}} %For explanation of the template
\newcommand{\an}[1]{\authornote{cyan}{Author}{#1}}

%% Common Parts

\newcommand{\progname}{ProgName} % PUT YOUR PROGRAM NAME HERE
\newcommand{\authname}{Team \#, Team Name
\\ Student 1 name
\\ Student 2 name
\\ Student 3 name
\\ Student 4 name} % AUTHOR NAMES                  

\usepackage{hyperref}
    \hypersetup{colorlinks=true, linkcolor=blue, citecolor=blue, filecolor=blue,
                urlcolor=blue, unicode=false}
    \urlstyle{same}
                                


\begin{document}

\title{Module Interface Specification for Centrality in Graphs}

\author{Atiyeh Sayadi}

\date{\today}

\maketitle

\pagenumbering{roman}

\section{Revision History}

\begin{tabularx}{\textwidth}{p{3cm}p{2cm}X}
\toprule {\bf Date} & {\bf Version} & {\bf Notes}\\
\midrule
March 12, 2024 & 1.0 & Initial draft\\
\bottomrule
\end{tabularx}

~\newpage

\section{Symbols, Abbreviations and Acronyms}

\renewcommand{\arraystretch}{1.2}
\begin{tabular}{l l} 
  \toprule		
  \textbf{symbol} & \textbf{description}\\
  \midrule 
  AC & Anticipated Change\\
CC & Closeness Centrality\\
  CIG& Centrality in Graphs\\
  DAG & Directed Acyclic Graph \\
DC & Degree Centrality\\
  M & Module \\
 n & Total number of nodes \\
  R & Requirement\\
  SRS & Software Requirements Specification\\
  UC & Unlikely Change \\
  \bottomrule
\end{tabular}\\

\newpage

\tableofcontents

\newpage

\pagenumbering{arabic}

\section{Introduction}

The following document details the Module Interface Specifications for Centrality in Graphs project. As stated earlier in various documents such as the SRS, this project aims to measure the centrality of each node in an undirected graph using two methods: degree centrality and closeness centrality.

Complementary documents include the System Requirement Specifications
and Module Guide.  The full documentation and implementation can be
found at \href{https://github.com/AtiyehSayadi/Centrality-In-Graphs/blob/main/docs/ProblemStatementAndGoals/ProblemStatement.pdf}{here}. \\
In general, the design steps for this project include the following:\\
\begin{enumerate}
\item Designing a module to obtain the graph structure, which is essentially the adjacency matrix. Then, using the constructed graph, metrics such as node degree and the length of the shortest path from one node to another can be obtained. The outputs of these two functions are used, respectively, for \href{https://towardsdatascience.com/graph-analytics-introduction-and-concepts-of-centrality-8f5543b55de3}{Degree Centrality} and \href{https://www.geeksforgeeks.org/closeness-centrality-centrality-measure/}{Closeness Centrality}.\\
\item Creating a module that reads the initial graph file and generates an output using the previously designed graph creation module, allowing a graphical user interface (GUI) to utilize it.\\
\item Additionally, the graphical user interface output needs to represent the obtained outputs in the form of an image, displaying a graph.
\end{enumerate}
In the design of this software, two modules, namely GUI Provider and Matrix, have been considered. Since these two modules utilize the libraries \href{https://docs.python.org/3/library/tk.html}{Tkinter} and \href{https://numpy.org/}{NumPy} respectively, serving as interfaces for these libraries, their interfaces will not be designed in this document.

\section{Notation}





The structure of the MIS for modules comes from \citet{HoffmanAndStrooper1995},
with the addition that template modules have been adapted from
\cite{GhezziEtAl2003}.  The mathematical notation comes from Chapter 3 of
\citet{HoffmanAndStrooper1995}.  For instance, the symbol := is used for a
multiple assignment statement and conditional rules follow the form $(c_1
\Rightarrow r_1 | c_2 \Rightarrow r_2 | ... | c_n \Rightarrow r_n )$.

The following table summarizes the primitive data types used by CIG. 

\begin{center}
\renewcommand{\arraystretch}{1.2}
\noindent 
\begin{tabular}{l l p{7.5cm}} 
\toprule 
\textbf{Data Type} & \textbf{Notation} & \textbf{Description}\\ 
\midrule
character & char & a single symbol or digit\\
integer & $\mathbb{Z}$ & a number without a fractional component in (-$\infty$, $\infty$) \\
natural number & $\mathbb{N}$ & a number without a fractional component in [1, $\infty$) \\
real & $\mathbb{R}$ & any number in (-$\infty$, $\infty$)\\
Graph& ${\mathbb{N}}^ {n\times n}$&\\

\bottomrule
\end{tabular} 
\end{center}

\noindent
The specification of CIG uses some derived data types: matrix. A matrix is a collection of elements arranged in rows and columns within a rectangular array. Each element in the matrix can be any scalar value, such as real numbers, complex numbers, or variables. Also, graph is a collection of nodes and edges, where each node represents elements within the network, and the edges represent the connections between them. In addition, CIG uses functions, which
are defined by the data types of their inputs and outputs. Local functions are
described by giving their type signature followed by their specification.

\section{Module Decomposition}

The following table is taken directly from the Module Guide document for this project.

\begin{table}[h!]
\centering
\begin{tabular}{p{0.3\textwidth} p{0.6\textwidth}}
\toprule
\textbf{Level 1} & \textbf{Level 2}\\
\midrule

{Hardware-Hiding Module} & - \\
\midrule

{Behaviour-Hiding Module} & Show Graph \\ & File\\ &Graph\\
\midrule

\multirow{3}{0.3\textwidth}{Software Decision Module} & Matrix\\ & GUI Provider\\
\bottomrule

\end{tabular}
\caption{Module Hierarchy}
\label{TblMH}
\end{table}

\newpage

\section{MIS of File } \label{Module} 


\subsection{Module}

File
\subsection{Uses}
Graph 
\subsection{Syntax}

\subsubsection{Exported Types}
N/A
\subsubsection{Exported Access Programs}

\begin{center}
\begin{tabular}{p{2cm} p{2cm} p{2cm} p{4cm}}
\hline
\textbf{Name} & \textbf{In} & \textbf{Out} & \textbf{Exceptions} \\
\hline
\texttt{read\_file} & string & GraphT & The file is empty. \\
\hline
\end{tabular}
\end{center}

\subsection{Semantics}

\subsubsection{State Variables}

N/A

\subsubsection{Environment Variables}

file: Set of tuple($\mathbb{N}$, $\mathbb{N}$)

\subsubsection{Assumptions}

N/A

\subsubsection{Access Routine Semantics}

\noindent \texttt{read\_file(s)}
\begin{itemize}
\item transition: - 
\item output: A graph of the Graph module to access the graph properties through it.
\item exception: When no file is uploaded or the file is empty, the function should issue a warning about the empty state of the variable "file".  
\end{itemize}

%\wss{A module without environment variables or state variables is unlikely to
  %have a state transition.  In this case a state transition can only occur if
  %the module is changing the state of another module.}

%\wss{Modules rarely have both a transition and an output.  In most cases you
  %will have one or the other.}

\subsubsection{Local Functions}

N/A

\newpage

\section{MIS of Graph} \label{Module} 


\subsection{Module}
-

\subsection{Uses}
Matrix
\subsection{Syntax}

\subsubsection{Exported Types}
GraphT
\subsubsection{Exported Access Programs}
\begin{center}
\begin{tabular}{p{6cm} p{2cm} p{2cm} p{6cm}}
\hline
\textbf{Name} & \textbf{In} & \textbf{Out} & \textbf{Exceptions} \\
\hline
\texttt{GraphT} & ${\mathbb{N}}$  & GraphT& - \\
\texttt{add\_edge} & $\mathbb{N}$& -&Inputs are out of range. \\
\texttt{exist\_edge} & $\mathbb{N}$, $\mathbb{N}$ & $\mathbb{B}$  & - \\
\texttt{get\_degree} & $\mathbb{N}$ & $\mathbb{N}$ & - \\
\texttt{get\_shortest\_path} & $\mathbb{N}$, $\mathbb{N}$ & $\mathbb{N}$ & - \\
\texttt{degree\_centrality} & $\mathbb{N}$ & $\mathbb{R}$ & - \\
\texttt{closeness\_centrality} & $\mathbb{N}$ & $\mathbb{R}$ & - \\

\hline
\end{tabular}
\end{center}

\subsection{Semantics}

\subsubsection{State Variables}

graph: ${\mathbb{N}}^ {n\times n}$ 


\subsubsection{Environment Variables}

N/A

\subsubsection{Assumptions}


{GraphT} is called before any other access programs.



\subsubsection{Access Routine Semantics}

\noindent \texttt{GraphT(g)}
\begin{itemize}
\item transition: 
\item output: A n*n zero matrix.
\item exception: -
\end{itemize}

\noindent \texttt{add\_edge(s, e)}
\begin{itemize}
\item transition: $graph[s,e]:=1,\ graph[e,s]:=1$
\item output: -
\item exception: If the indices are out of bounds, accessing the matrix is not possible, so in this case, an error alert is issued by the function.  
\end{itemize}

\noindent \texttt{exist\_edge(s, e)}
\begin{itemize}
\item transition: -
\item output: $out:= ( graph[s,e] ==1)$
\item exception:-  
\end{itemize}

\noindent \texttt{get\_degree(n)}
\begin{itemize}
\item transition: -
\item output: $out:= {\sum_{j=0}^{|graph|-1}{graph[n,j]}}$
\item exception: -
\end{itemize}

\noindent \texttt{get\_shortest\_path(s,n)}
\begin{itemize}
\item transition: -
\item output: $out:=Min(|s,a1,a2,..n|) \\The \ shortest \ path\  between\  nodes\  s\  and \ n .$
\item exception: -   
\end{itemize}

\noindent \texttt{degree\_centrality(n)}
\begin{itemize}
\item transition: -
\item output: $out:=\frac{get\_degree(n)}{|graph| - 1}  $
\item exception: - 
\end{itemize}

\noindent \texttt{closeness\_centrality(n)}
\begin{itemize}
\item transition: -
\item output: $out:= \frac{|graph|- 1}{\sum_{j=0}^{|graph|-1}{get\_shortest\_path(n,j)}} $
\item exception: - 
\end{itemize}

\subsubsection{Local Functions}
N/A
\newpage

\section{MIS Show Graph} \label{Module} 


\subsection{Module}
Show Graph

\subsection{Uses}
File
\subsection{Syntax}

\subsubsection{Exported Constants}
N/A
\subsubsection{Exported Access Programs}
\begin{center}
\begin{tabular}{|p{6cm}|p{2.5cm}|p{2.5cm}|p{4cm}|}
\hline
\textbf{Name} & \textbf{In} & \textbf{Out} & \textbf{Exceptions} \\
\hline
\texttt{show\_degree\_centrality} & GraphT& - & - \\
\texttt{show\_closeness\_centrality} & GraphT & - & - \\
\hline
\end{tabular}
\end{center}

\subsection{Semantics}

\subsubsection{State Variables}

N/A

\subsubsection{Environment Variables}

win: two-dimensional sequence of colored pixels

\subsubsection{Assumptions}

N/A

\subsubsection{Access Routine Semantics}

\noindent \texttt{degree\_centrality(g)}
\begin{itemize}
\item transition: Adjusting the pixels of the win for displaying the graph g for degree centrality
\item output: -
\item exception: N/A  
\end{itemize}

\noindent \texttt{closeness\_centrality(g)}
\begin{itemize}
\item transition: Adjusting the pixels of the win for displaying the graph g for closeness centrality
\item output: -
\item exception: N/A  
\end{itemize}

\subsubsection{Local Functions}

N/A

\newpage



\bibliographystyle {plainnat}
\bibliography {../../../refs/References}




\end{document}