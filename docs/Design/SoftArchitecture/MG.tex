\documentclass[12pt, titlepage]{article}

\usepackage{fullpage}
\usepackage[round]{natbib}
\usepackage{multirow}
\usepackage{booktabs}
\usepackage{tabularx}
\usepackage{graphicx}
\usepackage{float}
\usepackage{hyperref}
\hypersetup{
    colorlinks,
    citecolor=blue,
    filecolor=black,
    linkcolor=red,
    urlcolor=blue
}

\input{../../Comments}
\input{../../Common}

\newcounter{acnum}
\newcommand{\actheacnum}{AC\theacnum}
\newcommand{\acref}[1]{AC\ref{#1}}

\newcounter{ucnum}
\newcommand{\uctheucnum}{UC\theucnum}
\newcommand{\uref}[1]{UC\ref{#1}}

\newcounter{mnum}
\newcommand{\mthemnum}{M\themnum}
\newcommand{\mref}[1]{M\ref{#1}}

\begin{document}

\title{Module Guide for Centrality in Graphs} 
\author{Atiyeh Sayadi}
\date{\today}

\maketitle

\pagenumbering{roman}

\section{Revision History}

\begin{tabularx}{\textwidth}{p{3cm}p{2cm}X}
\toprule {\bf Date} & {\bf Version} & {\bf Notes}\\
\midrule
Date 1 & 1.0 & Notes\\
Date 2 & 1.1 & Notes\\
\bottomrule
\end{tabularx}

\newpage

\section{Reference Material}

This section records information for easy reference.

\subsection{Abbreviations and Acronyms}

\renewcommand{\arraystretch}{1.2}
\begin{tabular}{l l} 
  \toprule		
  \textbf{symbol} & \textbf{description}\\
  \midrule 
  AC & Anticipated Change\\
CC & Closeness Centrality\\
  CIG& Centrality in Graphs\\
  DAG & Directed Acyclic Graph \\
DC & Degree Centrality\\
  M & Module \\
  MG & Module Guide \\
  OS & Operating System \\
  R & Requirement\\
  SC & Scientific Computing \\
  SRS & Software Requirements Specification\\
  UC & Unlikely Change \\
  \bottomrule
\end{tabular}\\

\newpage

\tableofcontents

\listoftables

\listoffigures

\newpage

\pagenumbering{arabic}

\section{Introduction}

Decomposing a system into modules is a commonly accepted approach to developing
software.  A module is a work assignment for a programmer or programming
team~\citep{ParnasEtAl1984}.  We advocate a decomposition
based on the principle of information hiding~\citep{Parnas1972a}.  This
principle supports design for change, because the ``secrets'' that each module
hides represent likely future changes.  Design for change is valuable in SC,
where modifications are frequent, especially during initial development as the
solution space is explored.  

Our design follows the rules layed out by \citet{ParnasEtAl1984}, as follows:
\begin{itemize}
\item System details that are likely to change independently should be the
  secrets of separate modules.
\item Each data structure is implemented in only one module.
\item Any other program that requires information stored in a module's data
  structures must obtain it by calling access programs belonging to that module.
\end{itemize}

After completing the first stage of the design, the Software Requirements
Specification (SRS), the Module Guide (MG) is developed~\citep{ParnasEtAl1984}. The MG
specifies the modular structure of the system and is intended to allow both
designers and maintainers to easily identify the parts of the software.  The
potential readers of this document are as follows:

\begin{itemize}
\item New project members: This document can be a guide for a new project member
  to easily understand the overall structure and quickly find the
  relevant modules they are searching for.
\item Maintainers: The hierarchical structure of the module guide improves the
  maintainers' understanding when they need to make changes to the system. It is
  important for a maintainer to update the relevant sections of the document
  after changes have been made.
\item Designers: Once the module guide has been written, it can be used to
  check for consistency, feasibility, and flexibility. Designers can verify the
  system in various ways, such as consistency among modules, feasibility of the
  decomposition, and flexibility of the design.
\end{itemize}

The rest of the document is organized as follows. Section
\ref{SecChange} lists the anticipated and unlikely changes of the software
requirements. Section \ref{SecMH} summarizes the module decomposition that
was constructed according to the likely changes. Section \ref{SecConnection}
specifies the connections between the software requirements and the
modules. Section \ref{SecMD} gives a detailed description of the
modules. Section \ref{SecTM} includes two traceability matrices. One checks
the completeness of the design against the requirements provided in the SRS. The
other shows the relation between anticipated changes and the modules. Section
\ref{SecUse} describes the use relation between modules.

\section{Anticipated and Unlikely Changes} \label{SecChange}

This section lists possible changes to the system. According to the likeliness
of the change, the possible changes are classified into two
categories. Anticipated changes are listed in Section \ref{SecAchange}, and
unlikely changes are listed in Section \ref{SecUchange}.

\subsection{Anticipated Changes} \label{SecAchange}

This software, like other softwares, may undergo changes. These changes are as follows:

\begin{description}
\item[\refstepcounter{acnum} \actheacnum \label{acHardware}:] In the future, it is possible that this software will be developed in another environment such as NetLogo.
\item[\refstepcounter{acnum} \actheacnum \label{acInput}:] This software can read the graph matrix directly from the input file as an adjacency matrix.
\item[\refstepcounter{acnum} \actheacnum \label{computation}:]  In the future, this software can also perform computations on directed graphs.
\end{description}

\subsection{Unlikely Changes} \label{SecUchange}

The future changes of the software do not include the following items:

\begin{description}
\item[\refstepcounter{ucnum} \uctheucnum \label{ucIO}:] The input file will only be in the form of a text file.
\item[\refstepcounter{ucnum} \uctheucnum \label{measure}:] Other centrality measures will not be calculated by this software.
\end{description}

\section{Module Hierarchy} \label{SecMH}

This section provides an overview of the module design. Modules are summarized
in a hierarchy decomposed by secrets in Table \ref{TblMH}. The modules listed
below, which are leaves in the hierarchy tree, are the modules that will
actually be implemented.

\begin{description}
\item [\refstepcounter{mnum} \mthemnum \label{mHH}:] Hardware-Hiding Module
\item [\refstepcounter{mnum} \mthemnum \label{mHH}:] Behaviour-Hiding Module
\item [\refstepcounter{mnum} \mthemnum \label{mHH}:] Software Decision Module
\end{description}


\begin{table}[h!]
\centering
\begin{tabular}{p{0.3\textwidth} p{0.6\textwidth}}
\toprule
\textbf{Level 1} & \textbf{Level 2}\\
\midrule

{Hardware-Hiding Module} & - \\
\midrule

{Behaviour-Hiding Module} & GUI\\ & File\\
\midrule

\multirow{3}{0.3\textwidth}{Software Decision Module} & Degree\\ & Closeness\\
\bottomrule

\end{tabular}
\caption{Module Hierarchy}
\label{TblMH}
\end{table}

\section{Connection Between Requirements and Design} \label{SecConnection}

The design of the system is intended to satisfy the requirements developed in
the SRS. In this stage, the system is decomposed into modules. The connection
between requirements and modules is listed in Table~\ref{TblRT}.\\\\
\textbf{Functional requirments:}
\begin{description}
\item FR1: Degree centrality for each node must be calculated accurately.
\item FR2: Closeness centrality for each node must be calculated accurately.
\item FR3: Degree centrality must be calculated for all nodes.
\item FR4: Closeness centrality must be calculated for all nodes.\\
\end{description}

\textbf{Nonfunctional requirments:}
\begin{description}
\item NFR1: Accuracy
\item NFR2: Usability
\item NFR3: Maintainability
\end{description}


\section{Module Decomposition} \label{SecMD}

Modules are decomposed according to the principle of ``information hiding''
proposed by \citet{ParnasEtAl1984}. 
The following is an introduction to the modules and a brief explanation of each of them.


\subsection{Hardware Hiding Modules }
%\begin{description}
%\item[Secrets:]The data structure and algorithm used to implement the virtual
%  hardware.
%\item[Services:]Serves as a virtual hardware used by the rest of the
 % system. This module provides the interface between the hardware and the
 % software. So, the system can use it to display outputs or to accept inputs.
%\item[Implemented By:] OS
%\end{description}
\subsection{Behaviour-Hiding Module}



\subsubsection{File (M1)}

\begin{description}
\item[Secrets:] Text file containing a graph matrix.
\item[Services:] This module reads the initial graph matrix from a text file and calculates its adjacency matrix.
\item[Implemented By:] CIG
\item[Type of Module:] Data Access Module.
\end{description}

\subsubsection{GUI (M2)}
\begin{description}
\item[Secrets:] Graphical User Interface (GUI) for displaying outputs.
\item[Services:] This module utilizes the results of calculations from the Degree, Closeness, and File modules and displays the values separately on the nodes of the graph.
\item[Implemented By:] CIG
\item[Type of Module:] Graphical User Interface
\end{description}


\subsection{Software Decision Module}

\subsubsection{Closeness(M3)}
\begin{description}
\item[Secrets:] Calculating the shortest paths from each node to all other nodes and then computing the closeness centrality for each node
\item[Services:] Using the output obtained from the File module, which is the adjacency matrix for the graph, it is necessary to first calculate the shortest path for each node to all other nodes according to the formula of closeness centrality. Then, based on this calculation, closeness centrality is computed. The output of this module will be in the form of a matrix of size n * 2, where n is the number of nodes in the graph. In the output matrix, the centrality of each node will be displayed against each node.
  % Changes in these modules are more likely to be motivated by a desire to
  % improve performance than by externally imposed changes.
\item[Implemented By:] CIG
\end{description}

\subsubsection{Degree(M4)}
\begin{description}
\item[Secrets:] Calculating the degree of each node in the graph and then calculating the degree centrality for each node in the graph.
\item[Services:] This module first determines the degree of each node in the graph using the output of the File module. Then, with the degree of each node and the total sum of degrees of all nodes in the graph, it calculates the degree centrality for each node. The output of this module will be in the form of a matrix of size n * 2, where n is the number of nodes in the graph. In the output matrix, the centrality of each node will be displayed against each node.
  % Changes in these modules are more likely to be motivated by a desire to
  % improve performance than by externally imposed changes.
\item[Implemented By:] CIG
\end{description}
\section{Traceability Matrix} \label{SecTM}

This section shows two traceability matrices: between the modules and the
requirements and between the modules and the anticipated changes.

% the table should use mref, the requirements should be named, use something
% like fref
\begin{table}[H]
\centering
\begin{tabular}{p{0.2\textwidth} p{0.6\textwidth}}
\toprule
\textbf{Req.} & \textbf{Modules}\\
\midrule
FR1 & M4 \\
FR2 & M3\\
FR3 & M4, M1\\
FR4 & M3, M1\\
NFR1 & M1, M2, M3, M4\\
NFR2 & M1, M2, M3, M4\\
NFR3 & M1, M2, M3, M4\\
\bottomrule
\end{tabular}
\caption{Trace Between Requirements and Modules}
\label{TblRT}
\end{table}

\begin{table}[H]
\centering
\begin{tabular}{p{0.2\textwidth} p{0.6\textwidth}}
\toprule
\textbf{AC} & \textbf{Modules}\\
\midrule
AC1 & M1, M2, M3, M4\\
AC2 & M1\\
AC3 & M2\\
\bottomrule
\end{tabular}
\caption{Trace Between Anticipated Changes and Modules}
\label{TblACT}
\end{table}

\section{Use Hierarchy Between Modules} \label{SecUse}

Figure 1 illustrates a hierarchical view of the relationship between the modules.
\begin{figure}[h!]
\begin{center}
 \includegraphics[width=0.6\textwidth]{GRAPH}
\caption{Use hierarchy among modules}
\label{Fig_SystemContext} 
\end{center}
\end{figure}

\section{User Interfaces}

Designing the user interface for this software is only for displaying outputs, in such a way that the user, by clicking on each button, will be shown the graph associated with it. 
You can refer to the user interface image in Figure 2.

\begin{figure}[h!]
\begin{center}
 \includegraphics[width=0.6\textwidth]{GUI}
\caption{Graphical User Interface}
\label{Fig_SystemContext} 
\end{center}
\end{figure}

\section{Design of Communication Protocols}

\section{Timeline}


Table 4 displays the timeline for the development of each module and its respective developer.

\begin{table}[H]
\centering
\begin{tabular}{p{0.2\textwidth} p{0.4\textwidth}  p{0.4\textwidth}}
\toprule
 \textbf{Modules} & \textbf{Time} & \textbf{Responsible} \\
\midrule
File& 11-15 March& Atiyeh Sayadi\\
Closeness & 16-22 March& Atiyeh Sayadi\\
Degree & 23-30 March& Atiyeh Sayadi\\
GUI & 1-3 April& Atiyeh Sayadi\\
\bottomrule
\end{tabular}
\caption{Timeline}
\end{table}

\bibliographystyle {plainnat}
\bibliography{../../../refs/References}

\newpage{}

\end{document}